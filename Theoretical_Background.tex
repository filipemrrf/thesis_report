%%%%%%%%%%%%%%%%%%%%%%%%%%%%%%%%%%%%%%%%%%%%%%%%%%%%%%%%%%%%%%%%%%%%%%%%
%                                                                      %
%     File: Thesis_Background.tex                                      %
%     Tex Master: Thesis.tex                                           %
%                                                                      %
%     Author: Andre C. Marta                                           %
%     Last modified :  4 Mar 2024                                      %
%                                                                      %
%%%%%%%%%%%%%%%%%%%%%%%%%%%%%%%%%%%%%%%%%%%%%%%%%%%%%%%%%%%%%%%%%%%%%%%%

\chapter{Theoretical Background}
\label{chapter:background}

In this chapter, we cover the theoretical background necessary for understanding the subsequent material of this work. We start by discussing the well-posedness of Cauchy problems in systems of partial differential equations in section \ref{section:well-posedness}, which is a necessary condition for our models to have predictive power. We then introduce the 3+1 formalism in section \ref{section:3+1_formalism}, which allows us to separate spacetime into three-dimensional space and time, making it easier to study the dynamical evolution of gravitational fields. Next, we discuss hyperboloidal compactification in section \ref{}, a technique that enables us to study the asymptotic behavior of solutions at future null infinity. Finally, we present the wave equation and its cubic non linearity in the sections \ref{} and \ref{} respectively, which are the central objects of our study.


%%%%%%%%%%%%%%%%%%%%%%%%%%%%%%%%%%%%%%%%%%%%%%%%%%%%%%%%%%%%%%%%%%%%%%%%
\section{Well-Posedness of Cauchy Problems in Systems of PDEs}
\label{section:well-posedness}

When dealing with systems of partial differential equations, one generally works with initial value problems (also sometimes called \textit{Cauchy problems}), which consist of, given initial data $u(0,x^i) = f(x^i)$ at a time $t = 0$, calculate the solution $u(t,x^i)$ for a later time. These problems are very relevant in physics due to the predictive character of our models. However, for our systems to possess this predictive power, they must be \textit{well-posed}. \textit{Well-posedness} in a partial differential equation problem is the requirement that there be a unique solution that depends continuously, in some norm, on the given data for the problem. More rigorously, we say an initial value problem is well-posed if there exist constants $K$ and $\alpha$ such that
%
\begin{align}
 ||u(t,\, \cdot)||_{L^2} \leq Ke^{\alpha \, t} ||f||_{L^2},
\end{align}
%
where the initial data of the \textit{Cauchy problem} is given by $u(0,x^i) = f(x^i)$ and the $L^2$ norm is given by
%
\begin{align}
 ||g||_{L^2}^2 = \int_{\mathbb{R}^3} g^\dagger\, g \;dx\,dy\,dz.
\end{align}

Without this property, small changes in the given data may result in arbitrarily large changes in the solution or render it impossible to find a solution at all. Thus, we must make sure to formulate our field equations in such a way that they form a well-posed system of differential equations.

Let us consider a system of partial differential equations that can be written as
%
\begin{align}
    \partial_t u = A^p \partial_p u + B \, u,
    \label{eq:strong_hyp}
\end{align}
%
where $u$ is a state vector. We call $A^p$ the \textit{principal matrix} of the system (even though it is an abbreviation for three matrices, one for each spatial dimension) and the remaining terms on the right-hand side of our system of differential equations non-principal. Given an arbitrary unit spatial vector $s^i$, we define the \textit{principal symbol} of the system as
%
\begin{align}
 P^s \equiv A^s = A^p s_p.
\end{align}

If, for every unit spatial vector $s^i$, the \textit{principal symbol} of our system has real eigenvalues, our system is considered \textit{weakly hyperbolic}. Additionally, if our system is \textit{weakly hyperbolic}, for every unit spatial vector $s^i$, the principal symbol has a complete set of eigenvectors, and there exists a constant $K$, independent of $s^i$, such that
%
\begin{align}
 ||T_s|| + ||T_s^{-1}|| \leq K,
\end{align}
%
where $T_s$ has the eigenvectors of $P^s$ as columns and we have the usual definition of the matrix norm $||\cdot||$; we call our system \textit{strongly hyperbolic}. The components of the vector $v = T^{-1}_s u$ are called \textit{characteristic variables} in the $s^i$ direction, which satisfy advection equations with speeds equal to the eigenvectors of the principal symbol up to non-principal terms and derivatives transverse to the $s^i$ direction. These variables will be relevant when we discuss how \texttt{bamps} handles patch boundaries in section \ref{}.


Let us now consider a system of partial differential equations written as
\begin{align}
    \partial_t u = A^p \partial_p u + F(t,x^i),
\end{align}
%
where we only consider solutions on the half-space $x^1 = x\geq 0$, making it so that we have a boundary. We call $F$ the forcing term, which could be, for example, $F = B /, u$ as in equation \eqref{eq:strong_hyp}. In this initial value problem, we provide initial conditions for the domain $u(0,x^i)=f(x^i)$, and boundary conditions $L\, u(t,x^i) \stackrel{\wedge}{=} g(t,x^A)$, where the index $A$ denotes that the data depends only on $x^2 = y$ and $x^3=z$, with $L$ a matrix that will be described later, and $\stackrel{\wedge}{=}$ denotes equality on the boundary.

We say that our system is \textit{symmetric hyperbolic} if there exists a Hermitian positive definite symmetrizer $H$ such that $H A^p s_p$ is Hermitian for every spatial vector $s_p$. This property of a system is very relevant as it is related to \textit{strong well-posedness} when the boundaries of our system are \textit{maximally dissipative}. It is important to note that every \textit{symmetric hyperbolic} system is \textit{strongly hyperbolic}, even though the opposite is not true.

Saying that an initial value problem is \textit{strongly well-posed} means that we can bind the solution in the bulk and restrict it to the initial boundary data with the addition of growth caused by non-principal terms or the boundaries. That is, an initial value problem is \textit{strongly well-posed} if, for every time $T$, there exists a constant $K_T$ independent of the initial data and the forcing terms such that, for $0 \leq t \leq T$, we have
%
\begin{align}
 ||u(t,\cdot)||^{2}_\Sigma + \int_{0}^{t} ||u(t',\cdot)||^{2}_{\partial \Sigma} \; dt' \leq K_{T}^{2} \left( ||f|^{2}_\Sigma + \int_{0}^{t} \left(||F(t',\cdot)||^{2}_\Sigma + ||g(t',\cdot)||^{2}_{\partial \Sigma} \right) dt' \right),  
\end{align}
%
where $||\cdot||_\Sigma$ denotes the $L^2$ norm on the half-space, and $||\cdot||_{\partial \Sigma}$ denotes the $L^2$ norm on the boundary plane.

Since every \textit{symmetric hyperbolic} system is \textit{strongly hyperbolic}, there exists a matrix $T_x$ such that
%
\begin{align}
 T_x^{-1}P^x T_x = \Lambda_x = \begin{pmatrix}   \Lambda_x^I & 0\\   0 & \Lambda_x^{II}   \end{pmatrix}.
\end{align}

If we have $\Lambda_I > 0$ and $\Lambda_{II} < 0$, our system has \textit{maximally dissipative} boundaries. If those conditions are not met, our system has characteristic boundaries in the variables where the characteristic speed vanishes at the boundaries.

We can prove the \textit{well-posedness} of our initial value problem by defining the energy
%
\begin{align}
 E = \int_\Sigma \epsilon \; dV,  
\end{align}
%
where $\epsilon = u^\dagger H u$. By taking the time derivative of this energy and performing an integration by parts, we get
%
\begin{align}
  \partial_tE + c_1 \int_{\partial\Sigma} \left( u^\dagger H u \right) \; dS \leq \int_\Sigma \left( u^\dagger H F + F^\dagger H u \right) \; dV + c_2 \int_{\partial \Sigma} \left( g^\dagger H g\right) \; dS ,  
\end{align}
%
for some $c_1, c_2 > 0$, from which we get \textit{strong well-posedness}. It is important to note that \textit{maximally dissipative} boundary conditions only guarantee \textit{well-posedness} for \textit{symmetric hyperbolic} systems, not for \textit{strongly hyperbolic} ones.

%%%%%%%%%%%%%%%%%%%%%%%%%%%%%%%%%%%%%%%%%%%%%%%%%%%%%%%%%%%%%%%%%%%%%%%%
\section{The 3+1 Formalism}
\label{section:3+1_formalism}

One usually finds the Einstein field equations written in a fully covariant form, where there isn't a clear distinction between space and time. However, in numerical relativity, we are interested in studying the dynamical evolution of a gravitational field in "time". It is thus convenient to split spacetime into three-dimensional space and time. That way, we can provide spacelike initial conditions and obtain the subsequent evolution along our time coordinate. To perform that separation, we use the so-called 3+1 Formalism.

We start by considering a spacetime with the metric $g_{ab}$~. To maintain hyperbolicity in our evolution equations, we must restrict our spacetimes to be globally hyperbolic, meaning they possess a Cauchy surface. These globally hyperbolic spacetimes can be completely foliated in a way such that each three-dimensional slice is spacelike. Additionally, we can identify each three-dimensional hypersurface with the level set of a parameter $t$~, which we consider to be a universal time function. 

Given two adjacent hypersurfaces $\Sigma_t$~and $\Sigma_{t+dt}$~in a specific foliation, we can determine the geometry of the region of spacetime between them by studying the movement of observers moving along the direction normal to the hypersurfaces. We call those observers \textit{Eulerian}. Considering those observers, we define three quantities that can describe our region of interest: the three-dimensional metric $\gamma_{ij}$~(which measures proper distances within the hypersurfaces), the lapse function $\alpha$~(which is the lapse of proper time between both hypersurfaces measured by \textit{Eulerian observers}), and the shift vector $\beta^i$~(which is the relative velocity between the \textit{Eulerian observers} and the lines of constant spatial coordinates). It is also useful to define the extrinsic curvature $K_{\mu\nu}$~(which measures the change of the normal vector to the hypersurface when parallel transported from one point in the hypersurface to another) and the acceleration of the \textit{Eulerian observers} $a^i$.

\begin{figure}[h]
\centering
\includegraphics[width=0.5\textwidth]{Figures/Definitions_of_the_3+1_quantities.png}
\caption{Geometric definition of the lapse function $\alpha$ and the shift vector $\beta^i$. The lapse function $\alpha$ is the proper time between two adjacent hypersurfaces $\Sigma_t$ and $\Sigma_{t+dt}$ measured by \textit{Eulerian observers}. The shift vector $\beta^i$ is the relative velocity between the \textit{Eulerian observers} and the lines of constant spatial coordinates.}
\end{figure}

In this work, we are interested in specifying initial conditions starting from a known spacetime metric $g_{ab}$~. Thus, we need to calculate the previously mentioned 3+1 quantities from $g_{ab}$~, which can be done by performing a 3+1 split of $g_{ab}$~. This split is done by writing $g_{ab}$~as a function of $\alpha$~, $\beta^i$~and $\gamma_{ij}$~:
%
\begin{align}
    ds^2 = (-\alpha^2 + \beta_i \beta^i) ~ dt^2 + 2 \beta_i ~ dt dx^i + \gamma_{ij} ~ dx^i dx^j,
\end{align}
%
or, more explicitly,
%
\begin{align}
    \begin{aligned}
        g_{\mu\nu} = \begin{pmatrix} -\alpha^2 + \beta_k \beta^k & \beta_i \\ \beta_j & \gamma_{ij} \end{pmatrix}, & \quad \quad \quad g^{\mu\nu} = \begin{pmatrix} -1/\alpha^2 & \beta^i/\alpha^2 \\ \beta^j/\alpha^2 & \gamma^{ij} - \beta^i \beta^j/\alpha^2\end{pmatrix}.
    \end{aligned}
\end{align}

Using this split, it becomes easy to obtain $\alpha$~, $\beta^i$~and $\gamma_{ij}$~from $g_{ab}$~, from which we can obtain the extrinsic curvature $K_{\mu\nu}$~using the fact that
%
\begin{align}
    K_{ij} = \frac{D_i \beta_j + D_j \beta_i - \partial_t \gamma_{ij}}{2 \alpha},
\end{align}
%
where $D_\mu := (\delta^\alpha_\mu + n^\alpha n_\mu) \nabla_\alpha$~is the projection of the covariant derivative onto the hypersurface and $n$~is the normal vector to the hypersurface, with components given by
%
\begin{align}
    \begin{aligned}
        n^\mu = (1/\alpha,~-\beta^i/\alpha), & \quad \quad \quad n_\mu = (-\alpha,~0).
    \end{aligned}
\end{align}

We can also obtain the acceleration of the \textit{Eulerian observers} $a^i$~as
\begin{align}
    a^i = \gamma^{ij} \partial_j(\ln \alpha).
\end{align}


%%%%%%%%%%%%%%%%%%%%%%%%%%%%%%%%%%%%%%%%%%%%%%%%%%%%%%%%%%%%%%%%%%%%%%%%
\section{Hyperboloidal Compactification}
\label{section:compactification}

In this work, we are interested in studying how waves behave at future null infinity, and thus it is useful to compactify our spacetime in such a way that we can study the asymptotic behavior of our solutions numerically. To do this, we will employ a specific set of coordinates called \textit{hyperboloidal coordinates}. 

The hyperboloidal coordinates $(t, r)$ are related to the usual spherical coordinates $(T, R)$ by the \textit{height function} $H(R)$~and the \textit{compress function} $\Omega(r)$~as follows:
%
\begin{align}
    T = t + H(R), \quad \quad \quad R = \Omega^{-1}(r) \, r.
\end{align}
%
The angular coordinates remain unchanged. This transformation gives rise to the Jacobian matrix
%
\begin{align}
    \left(J^{Hyp}\right)_{\alpha'}^{\ \ \beta} = 
    \begin{pmatrix}
        1 & -H'(r) & 0 & 0 \\
        0 & \frac{L(r)}{\Omega^2(r)} & 0 & 0 \\
        0 & 0 & 1 & 0 \\
        0 & 0 & 0 & 1
    \end{pmatrix} \; ,
\end{align}
%
where $H'(r)$ denotes the derivative of the height function with respect to $R$ written as a function of $r$, and $L(r)$ is defined as
%
\begin{align}
    L(r) \equiv \Omega(r) - r \, \partial_r \Omega(r) \; .
\end{align}

It is important to note that, even though there is freedom to choose the height and the compress function, we must require that, assymptotically in $R$, or equivalently, as $\Omega$ approaches zero, we have
%
\begin{align}
    1 - H' \sim O(\Omega^2) \, .
\end{align}

******* Write how these transformations affect minkowski metric *******

******* Write about layers *******

Since the solutions to our equations decay as we approach null infinity, it is useful to rescale our spacetime in such a way that our solutions are of order one all through space. Given a spacetime with metric $g_{ab}$~, we can rescale it by a strictly positive, smooth scalar function $\Omega$~to obtain a new metric $\tilde{g}_{ab}$~:
%
\begin{align}
    \tilde{g}_{ab} 
    = \Omega^2 g_{ab},
\end{align}
%
This rescaling is called a \textit{conformal transformation} and the function $\Omega$~ is called the \textit{conformal factor}. If $g_{ab}$~is a Lorentzian metric, then a vector $v^a$~is timelike, null or spacelike with respect to $g_{ab}$~ if and only if it is timelike, null or spacelike with respect to $\tilde{g}_{ab}$~, meaning that the causal structure of the spacetime is preserved under conformal compactification. 

Let us consider a scalar field $\psi$ on a spacetime with metric $g_{ab}$~. The wave equation for $\psi$~ is given by
%
\begin{align}
    \Box_g \psi 
    = g^{ab} \nabla_a \nabla_b \psi 
    = S,
    \label{eq:wave_equation}
\end{align}

%
where $S$~is a source term. If we perform a conformal transformation of the spacetime, the d'Alembert operator in the new metric $\tilde{g}_{ab}$~becomes
%
\begin{align}
    \Box_{\tilde{g}} \tilde{\psi} 
    = \tilde{g}^{ab} \tilde{\nabla}_a \tilde{\nabla}_b \tilde{\psi} 
    =&~\Omega^{s-2} g^{ab} \nabla_a \nabla_b \psi 
    + (2s + n - 2) \Omega^{s-3} g^{ab} \nabla_a \Omega \nabla_b \psi \nonumber \\
    &+ s \Omega^{s-3} \psi\, g^{ab} \nabla_a \nabla_b \Omega 
    + s(n + s - 3) \Omega^{s-4} \psi\, g^{ab} \nabla_a \Omega \nabla_b \Omega,
    \label{eq:conformal_wave_equation1}
\end{align}
%
where $s$~is the conformal weight of the scalar field $\psi$ (which we can choose freely)~and $n$~is the dimension of the spacetime. In this work, we will be working in a four-dimensional spacetime, so $n = 4$~. The conformal weight $s$~will be chosen to be $s = -1$ to simplify the expression of equation \eqref{eq:conformal_wave_equation1}. Thus, we have
%
\begin{align}
    \tilde{g}^{ab} \tilde{\nabla}_a \tilde{\nabla}_b \tilde{\psi} 
    =&~\Omega^{-3} g^{ab} \nabla_a \nabla_b \psi 
    - \Omega^{-4} \psi\, g^{ab} \nabla_a \nabla_b \Omega,
    \label{eq:conformal_wave_equation2}
\end{align}
%
We can now substitute \eqref{eq:wave_equation} into \eqref{eq:conformal_wave_equation2} and write the resulting equation in terms of the rescaled quantities, obtaining
%
\begin{align}
    \tilde{g}^{ab} \tilde{\nabla}_a \tilde{\nabla}_b \tilde{\psi} 
    =&~\Omega^{-3} S 
    - \Omega^{-1} \tilde{\psi} \bigg( 
        \tilde{g}^{ab} \tilde{\nabla}_a \tilde{\nabla}_b \Omega 
        - 2 \Omega^{-1} \tilde{g}^{ab} \tilde{\nabla}_a \Omega \tilde{\nabla}_b \Omega \bigg),
\end{align}

%
which is the wave equation for the rescaled scalar field $\tilde{\psi}$~in the compactified spacetime with metric $\tilde{g}_{ab}$~.

%%%%%%%%%%%%%%%%%%%%%%%%%%%%%%%%%%%%%%%%%%%%%%%%%%%%%%%%%%%%%%%%%%%%%%%%
\section{The Wave Equation}
\label{section:wave_equation}

The wave equation is an object of great interest in physics, as it describes the propagation of all types wave-like phenomena ranging from the waves of a beach to gravitational waves (as long as the correct modifications are made). In the study of gravitational waves, this equation is of particular importance due to its structural similarity to the Einstein field equations on the \textit{Generalized Harmonic Gauge (GHG)}. The wave equation, in its simpler form, is given by
%
\begin{align}
    \Box \Psi = S,
    \label{eq:wave_equation_simple}
\end{align}
%
where $\Box$ is the d'Alembert operator, $\Psi$ is the scalar field, and $S$ is a source term. The Einstein equations in the GHG can be written as
%
\begin{align}
    \Box_g g_{ab} = 2 g^{cd} g^{ef} (\partial_e g_{ca} \partial_f g_{db} - \Gamma_{ace} \Gamma_{bdf}),
\end{align}
%
where $g_{ab}$ is the spacetime metric, $\Gamma_{abc}$ are the Christoffel symbols, and $g^{ab}$ are the components of the inverse metric. The similarity between these two equations allows us to study the wave equation as a model for gravitational waves.

Let us now consider the wave equation in equation \eqref{eq:wave_equation_simple} without a source. The conserved energy for this equation is given by
%
\begin{align}
    \mathcal{E}(\Psi) = \frac{1}{2} \int \, \left( \partial_t \Psi(t,r) + \partial_r \Psi(t,r) \right) d^4x,
\end{align}


%%%%%%%%%%%%%%%%%%%%%%%%%%%%%%%%%%%%%%%%%%%%%%%%%%%%%%%%%%%%%%%%%%%%%%%%
\section{The Cubic Wave Equation}
\label{section:cubic_wave_equation}