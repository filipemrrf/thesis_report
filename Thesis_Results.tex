%%%%%%%%%%%%%%%%%%%%%%%%%%%%%%%%%%%%%%%%%%%%%%%%%%%%%%%%%%%%%%%%%%%%%%%%
%                                                                      %
%     File: Thesis_Results.tex                                         %
%     Tex Master: Thesis.tex                                           %
%                                                                      %
%     Author: Andre C. Marta                                           %
%     Last modified :  4 Mar 2024                                      %
%                                                                      %
%%%%%%%%%%%%%%%%%%%%%%%%%%%%%%%%%%%%%%%%%%%%%%%%%%%%%%%%%%%%%%%%%%%%%%%%

\chapter{Results}
\label{chapter:results}

Insert your chapter material here.


%%%%%%%%%%%%%%%%%%%%%%%%%%%%%%%%%%%%%%%%%%%%%%%%%%%%%%%%%%%%%%%%%%%%%%%%
\section{Wave Equation in Hyperboloidal Slices}
\label{section:hyp_wave}

The first step towards our goal is to implement the wave equation from equation \eqref{eq:wave_equation_3+1} in hyperboloidal slices. We will construct our hyperboloidal slices using 
%
\begin{align}
    H(R) = \frac{2 R^2 + s^2 - \sqrt{4 R^2 s^2 + s^4}}{2 \sqrt{R^2 + 1}} \,, \quad \quad \quad \Omega(r) = 1 - \frac{r^2}{s^2} \,,
\end{align}
%
as heigh and compress functions respectively, where $s$ is a free parameter that controls where null infinity is located. The details of this construction were discussed in section \ref{section:compactification}. This particular choice of heigh and compress functions is of particular interest since it leads to a constant outgoing characteristic speed, which has the advantage of keeping the shape of the waves as they propagate \cite{Hyperboloidal_layers_for_hyperbolic_equations_on_unbounded_domains,The_evolution_of_hyperboloidal_data_with_the_dual_foliation_formalism_Mathematical_analysis_and_wave_equation_tests}. This, however comes with the disadvantage of having an incoming characteristic speed that goes to zero as we approach null infinity. This is an acceptable trade-off since we are not expecting incoming waves in our setup. In figure \ref{fig:hyp_speeds} we can see how the incoming and outgoing light speeds vary along our computational domain.

\textcolor{red}{INSERT FIGURE WITH SPEEDS HERE}

Giving the initial conditions 
%
\begin{align}
    sjdjfçlsdg
\end{align}
%
and setting $s = 30$, we obtain the evolution shown in figures \ref{fig:hyp_wave1} and \ref{fig:hyp_wave2}, which satisfies the 


%%%%%%%%%%%%%%%%%%%%%%%%%%%%%%%%%%%%%%%%%%%%%%%%%%%%%%%%%%%%%%%%%%%%%%%%
\section{Wave Equation in Hyperboloidal Layers}
\label{section:hyp_layers_wave}




%%%%%%%%%%%%%%%%%%%%%%%%%%%%%%%%%%%%%%%%%%%%%%%%%%%%%%%%%%%%%%%%%%%%%%%%
\section{Cubic Wave Equation}
\label{section:hyp_cubic_wave}



